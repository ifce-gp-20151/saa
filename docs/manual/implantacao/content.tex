\section{Introdução}\label{introduuxe7uxe3o}

Este documento visa mostrar como o sistema SAA deve ser instalado e
configurado para ambiente de produção.

\section{Sistema Operacional}\label{sistema-operacional}

Debian 7 wheezy (\href{http://debian.c3sl.ufpr.br/debian-cd/}{Download})
ou Ubuntu Server 14.04 LTS
(\href{http://www.ubuntu.com/download/server}{Download})

\textbf{Obs.}: Todo o restante da documentação é baseada nestes sistema,
pois ambos utilizam a mesma tecnologia para instalação de pacotes.

\section{Banco de dados}\label{banco-de-dados}

O SAA utiliza \href{http://www.postgresql.org/}{PostgreSQL} como banco
de dados relacional.

Última versão estável: \texttt{9.4}.

\subsection{Debian}\label{debian}

\url{http://www.postgresql.org/download/linux/debian/}

Crie o arquivo \texttt{/etc/apt/sources.list.d/pgdg.list}, adicione a
linha para o repositório:

\begin{verbatim}
deb http://apt.postgresql.org/pub/repos/apt/ wheezy-pgdg main
\end{verbatim}

Importe a chave assinada do repositório, e atualize a lista de pacotes

\begin{verbatim}
wget --quiet -O - https://www.postgresql.org/media/keys/ACCC4CF8.asc | \
  sudo apt-key add -
sudo apt-get update
apt-get install postgresql-9.4
\end{verbatim}

\subsection{Ubuntu}\label{ubuntu}

\url{http://www.postgresql.org/download/linux/ubuntu/}

Crie o arquivo \texttt{/etc/apt/sources.list.d/pgdg.list}, adicione a
linha para o repositório:

\begin{verbatim}
deb http://apt.postgresql.org/pub/repos/apt/ trusty-pgdg main
\end{verbatim}

Importe a chave assinada do repositório, e atualize a lista de pacotes

\begin{verbatim}
wget --quiet -O - https://www.postgresql.org/media/keys/ACCC4CF8.asc | \
  sudo apt-key add -
sudo apt-get update
apt-get install postgresql-9.4
\end{verbatim}

\section{Obtendo código-fonte}\label{obtendo-cuxf3digo-fonte}

O projeto está disponível em \url{https://github.com/ifce-gp-20151/saa}.

Para a instalação em ambiente de produção é necessário baixar um
\emph{release} estável e testado. Os \emph{releases} podem ser baixados
em \url{https://github.com/ifce-gp-20151/saa/releases}.

\subsection{Instalando dependências}\label{instalando-dependuxeancias}

Das dependências descritas em \texttt{composer.json}, são necessárias:

\begin{itemize}
\itemsep1pt\parskip0pt\parsep0pt
\item
  zendframework/zendframework
\item
  doctrine/doctrine-orm-module
\item
  dino/dompdf-module
\end{itemize}

Subir para o servidor somente o necessário para o funcionamento.

Em \texttt{config/application.config.php} comentar as linhas referentes
aos \texttt{modules}:

\begin{itemize}
\itemsep1pt\parskip0pt\parsep0pt
\item
  ZFTool
\item
  ZendDeveloperTools
\end{itemize}

\subsection{Configuração da conexão com banco de
dados}\label{configurauxe7uxe3o-da-conexuxe3o-com-banco-de-dados}

Em \texttt{config/autoload/global.php}, modificar trecho:

\begin{verbatim}
'db' => array(
    'driver' => 'Pdo',
    'dsn' => 'pgsql:host=localhost;dbname=saa_production',
),
\end{verbatim}

Em seguida criar arquivo \texttt{config/autoload/local.php}, colocando
os dados:

\begin{verbatim}
'db' => array(
    'username' => 'postgres',
    'password' => 'secret',
),
\end{verbatim}

\subsection{Scripts de ACL}\label{scripts-de-acl}

\emph{Em desenvolvimento\ldots{}}

\subsection{Merge de release anterior}\label{merge-de-release-anterior}

\emph{Em desenvolvimento\ldots{}}

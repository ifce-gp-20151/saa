\section{Sistema recomendado}\label{sistema-recomendado}

\href{http://www.ubuntu.com/download/desktop/}{Ubuntu 14.04 LTS}

\subsection{Dependências}\label{dependuxeancias}

Execute o \emph{script} abaixo para instalar as dependências:

\begin{verbatim}
sudo add-apt-repository ppa:webupd8team/java
sudo apt-get update
sudo apt-get install oracle-java7-installer
sudo apt-get install php5 apache2 php5-pgsql php5-mcrypt php5-intl
\end{verbatim}

\subsection{IDE}\label{ide}

Escolha a IDE de sua preferência.

Recomendado: \href{https://netbeans.org/}{NetBeans IDE}.

Outras:

\begin{itemize}
\itemsep1pt\parskip0pt\parsep0pt
\item
  \href{https://atom.io/}{Atom}
\item
  \href{http://eclipse.org/pdt/}{Eclipse IDE}
\end{itemize}

\subsection{PostgreSQL}\label{postgresql}

\url{http://www.postgresql.org/download/linux/ubuntu/}

Crie o arquivo \texttt{/etc/apt/sources.list.d/pgdg.list}, adicione a
linha para o repositório:

\begin{verbatim}
deb http://apt.postgresql.org/pub/repos/apt/ trusty-pgdg main
\end{verbatim}

Importe a chave assinada do repositório, e atualize a lista de pacotes

\begin{verbatim}
wget --quiet -O - https://www.postgresql.org/media/keys/ACCC4CF8.asc | \
  sudo apt-key add -
sudo apt-get update
apt-get install postgresql-9.4
\end{verbatim}

\section{Configurar projeto local}\label{configurar-projeto-local}

Faça um \emph{clone} do projeto usando ssh:

\begin{verbatim}
cd NetBeansProjects
git clone git@github.com:ifce-gp-20151/saa.git
\end{verbatim}

\textbf{Obs.}: Você pode utilizar outro diretório para seu projeto, não
é necessário ser \texttt{NetBeansProjects}.

\textbf{Obs.}: Se você ainda não configurou sua chave ssh acesse este
\href{https://help.github.com/articles/generating-ssh-keys/}{link}.

Entre no diretório do projeto:

\begin{verbatim}
cd saa
\end{verbatim}

Instale as dependências do projeto:

\begin{verbatim}
php composer.phar install
\end{verbatim}

Crie o arquivo \texttt{local.php} em \texttt{config/autoload} com o
conteúdo:

\begin{verbatim}
<?php

return array(
    'db' => array(
        'username' => 'postgres',
        'password' => '**secret**',
    )
);
\end{verbatim}

Crie o arquivo \texttt{ocra-service-manager.local.php} em
\texttt{config/autoload} com o conteúdo:

\begin{verbatim}
<?php

return array(
    'ocra_service_manager' => array(
        // Turn this on to disable dependencies view in Zend Developer Tools
        'logged_service_manager' => true,
    ),
);
\end{verbatim}

Copie o arquivo \texttt{zenddevelopertools.local.php.dist}:

\begin{verbatim}
cp ./vendor/zendframework/zend-developer-tools/config/zenddevelopertools.local.php.dist \
./config/autoload/zenddevelopertools.local.php
\end{verbatim}

Execute os scripts abaixo usando PgAdmin.

\begin{itemize}
\itemsep1pt\parskip0pt\parsep0pt
\item
  \texttt{./docs/db/ddl.acl.sql}
\item
  \texttt{./docs/db/dml.acl.sql}
\end{itemize}

\subsubsection{Configuração Apache 2
(2.2.22-1ubuntu1.7)}\label{configurauxe7uxe3o-apache-2-2.2.22-1ubuntu1.7}

Crie o arquivo \texttt{saa} em \texttt{/etc/apache2/sites-enabled} com o
conteúdo:

\begin{verbatim}
<VirtualHost *:80>
    ServerName saa.local
    DocumentRoot /path/to/saa/public

    <Directory /path/to/saa/public>
        DirectoryIndex index.php
        AllowOverride All
        Order allow,deny
        Allow from all
    </Directory>
</VirtualHost>
\end{verbatim}

\subsubsection{Apache 2
(2.4.7-1ubuntu4.1)}\label{apache-2-2.4.7-1ubuntu4.1}

Crie o arquivo \texttt{saa.conf} em \texttt{/etc/apache2/sites-enabled}
com o conteúdo:

\begin{verbatim}
<VirtualHost *:80>
    ServerName saa.local

    DocumentRoot /path/to/saa/public

    <Directory /path/to/saa/public/>
        Options Indexes FollowSymLinks
        AllowOverride All
        Require all granted
    </Directory>

    ErrorLog /var/log/apache2/error.log
    CustomLog /var/log/apache2/access.log combined
</VirtualHost>
\end{verbatim}

Reinicie o apache2:

\begin{verbatim}
sudo service apache2 restart
\end{verbatim}

Crie um virtual host, edite \texttt{/etc/hosts}, e adicione esta linha
ao final:

\begin{verbatim}
127.0.1.1       saa.local
\end{verbatim}

Ok, você deve estar pronto agora. Acesse o endereço
\url{http://saa.local}

\subsection{Armadilhas}\label{armadilhas}

\subsubsection{Erro de sessão}\label{erro-de-sessuxe3o}

\url{http://stackoverflow.com/questions/26377753/zend-authentication-storage-session-session-validation-failed-any-ideas}

O ZendDevelopersTools possui uma função que quebra o sistema. O link
acima descreve o problema.

\paragraph{Solução}\label{soluuxe7uxe3o}

Comente a linha 64 do arquivo:

\texttt{./vendor/zendframework/zend-developer-tools/src/ZendDeveloperTools/Listener/EventLoggingListenerAggregate.php}.

\begin{verbatim}
public function attachShared(SharedEventManagerInterface $events)
{
    //$events->attach($this->identifiers, '*', array($this, 'onCollectEvent'), Profiler::PRIORITY_EVENT_COLLECTOR);
}
\end{verbatim}

\subsection{Criação de entidades (Entities) a partir do banco de
dados}\label{criauxe7uxe3o-de-entidades-entities-a-partir-do-banco-de-dados}

\begin{verbatim}
./vendor/bin/doctrine-module orm:convert-mapping --filter="Usuario" \
--from-database annotation --namespace="Application\\Entity\\" \
module/Application/src
\end{verbatim}

onde \texttt{Usuario} é o nome da tabela a ser gerada.

\subsection{Criação dos \emph{getters} e
\emph{setters}}\label{criauxe7uxe3o-dos-getters-e-setters}

\begin{verbatim}
./vendor/bin/doctrine-module orm:generate-entities \
--filter="Usuario" module/Application/src/
\end{verbatim}

\subsection{Links úteis}\label{links-uxfateis}

\begin{itemize}
\itemsep1pt\parskip0pt\parsep0pt
\item
  \url{http://ocramius.github.io/presentations/doctrine2-zf2-introduction/\#/1}
\end{itemize}

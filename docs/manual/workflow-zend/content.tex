\section{Definir Autorização (ACL)}\label{definir-autorizauxe7uxe3o-acl}

No arquivo \texttt{./docs/db/dml.acl.sql} adicione os \texttt{INSERT}'s
necessários.

Exemplo para um \emph{Controller} \texttt{Professor} no módulo
\texttt{Application}, com a \emph{action} \texttt{index}, em que apenas
o \texttt{admin} tem acesso.

Neste arquivo temos as \emph{Roles} ou Papeis, indicando qual o tipo de
usuário. Algumas \emph{roles} já foram definidas, como a \texttt{admin}.

Temos também os \emph{Modules}. O módulo que a gente quer também já
existe \texttt{Application}.

Em seguida temos os \texttt{Controllers}. Esse vamos precisar criar.
Para isso basta adicionar uma linha, colocando o id como incremento do
anterior e o nome do controller:

\begin{verbatim}
INSERT INTO saa.acl_controllers(id, controller) VALUES ( 9, 'Professor');
\end{verbatim}

Abaixo temos as \emph{Actions}. A action \texttt{index} também já
existe.

Em seguida temos os \emph{Resources} ou Recursos, indicando o trio
\emph{Module}, \emph{Controller}, \emph{Action}. Crie a linha:

\begin{verbatim}
INSERT INTO saa.acl_resources(id, module_id, controller_id, action_id)
VALUES (28, 1, 9, 1);--application/professor/index
\end{verbatim}

Por fim temos os \emph{Privileges} ou Privilégios, indicando que
\emph{Role} tem acesso a que \emph{Resource}. Crie no local indicado
pelos comentários:

\begin{verbatim}
-- 
-- admin
-- 
INSERT INTO saa.acl_privileges(resource_id, role_id, allow)
VALUES ( 28, 3, true);--application/professor/index
\end{verbatim}

Execute as queries no banco de dados.

\section{Crie uma rota}\label{crie-uma-rota}

As rotas são definidas no arquivo
\texttt{./module/Application/config/module.config.php}. Existe um
arquivo desses para cada módulo. Encontre o trecho:

\begin{verbatim}
'router' => array(
    'routes' => array(
        /* código existente ... */
    ),
),
\end{verbatim}

Adicione a linha:

\begin{verbatim}
'professor' => array(
    'type' => 'Segment',
    'options' => array(
        'route'    => '/professor',
        'defaults' => array(
            'controller' => 'Application\Controller\Professor',
            'action'     => 'index',
            'module'     => 'application',
        ),
    ),
),
\end{verbatim}

\section{Mapeie o Controller}\label{mapeie-o-controller}

Os controllers são mepeados nesse mesmo arquivo. Encontre o trecho:

\begin{verbatim}
'controllers' => array(
        'invokables' => array(
            /* código existente ... */
        ),
    ),
\end{verbatim}

Adicione a linha:

\begin{verbatim}
'Application\Controller\Professor' => 'Application\Controller\ProfessorController',
\end{verbatim}

\section{Criar o Controller}\label{criar-o-controller}

Crie o arquivo \texttt{ProfessorController} dentro de
\texttt{./module/Application/src/Application/Controller/}.

Com o conteúdo:

\begin{verbatim}
<?php

namespace  Application\Controller;

use Zend\View\Model\ViewModel;
use Core\Controller\ActionController;

class ProfessorController extends ActionController {

    /**
     * @var DoctrineORMEntityManager
     */
    protected $em;
    
    private function getEntityManager() {
        if (null === $this->em) {
            $this->em = $this->getServiceLocator()
                ->get('doctrine.entitymanager.orm_default');
        }
        return $this->em;
    }
    
    public function indexAction() {
        return new ViewModel(array());
    }
}
\end{verbatim}

\section{Criar View}\label{criar-view}

Crie o arquivo \texttt{index.phtml} dentro de

\texttt{./module/Application/view/application/professor/}.

Com o conteúdo:

\begin{verbatim}
<h3>você está em application/professor/index</h3>
\end{verbatim}

\section{Acessar Action}\label{acessar-action}

Faça login com usuário da \emph{role} \texttt{admin}, digite no final da
URL: \texttt{/professor}.
